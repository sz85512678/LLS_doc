\documentclass[english,11pt]{article}

%PUT ALL THE PREAMBLES HERE
\usepackage[T1]{fontenc}
\usepackage[latin9]{inputenc}

% \makeatletter % to be able to overwrite package containing @
\usepackage{siunitx}
% \usepackage{subfig}
\usepackage{colonequals}
\usepackage[super]{nth}
\usepackage{algorithm}
\usepackage{amsmath}
\usepackage{amsthm}
\usepackage{amssymb}
\usepackage{tikz}
\usepackage{url}
\usepackage{float}
\usepackage{tabularx}
\usepackage{pbox}
\usepackage{tabulary}
\usepackage{breqn}
\usepackage{graphicx}
\usepackage{wrapfig}
\usepackage[font={small}]{caption}
\usepackage{enumitem}
\usepackage{physics}
\usepackage{lipsum}
\usepackage{mwe}
\usepackage{breqn}
\usepackage[noend]{algpseudocode}
\usepackage{subcaption}
\usepackage{graphicx}
\usepackage{enumitem}

\usepackage{hyperref}
\hypersetup{
    colorlinks,
    citecolor=black,
    filecolor=black,
    linkcolor=black,
    urlcolor=black
} % set up hyperlink for table of content
% \usepackage[]{algorithm2e}
% \SetKwInOut{Parameter}{Parameter}

\graphicspath{{./images/}{./}}

\newtheorem{claim}{Claim}
\newtheorem{theorem}{Theorem}
\newtheorem{lemma}{Lemma}
\newtheorem{remark}{Remark}
\newtheorem{corallary}{Corollary}
\newtheorem{conjecture}{Conjecture}
\newtheorem*{observation}{Observation}
\newtheorem*{myQuote}{Quote}
\newtheorem{fact}{Fact}
\newtheorem{definition}{Definition}
\newtheorem{assumption}{Assumption}
\newtheorem{example}{Example}
% \newtheorem{proof}{Proof}

\newcommand{\R}{\mathbb{R}}
\newcommand{\E}{\mathbb{E}}
\renewcommand{\P}{\mathbb{P}}
\newcommand{\I}{\mathbb{I}}

% \renewenvironment{proof}{{\bfseries Proof}}{\qed}

\makeatletter
\def\BState{\State\hskip-\ALG@thistlm}
\makeatother

\newtheorem*{theorem*}{Theorem}
\newcommand*{\vertbar}{\rule[-1ex]{0.5pt}{2.5ex}}
\newcommand*{\horzbar}{\rule[.5ex]{2.5ex}{0.5pt}}

\newcommand{\norms}[1]{\|#1\|_2^2}

\DeclareMathOperator*{\argmax}{arg\,max}
\DeclareMathOperator*{\argmin}{arg\,min}
\newcommand{\bracket}[1]{\left( #1\right)}
\newcommand{\squareBracket}[1]{\left[ #1 \right]}

\renewcommand{\norms}[1]{\|#1\|_2^2}
\newcommand{\la}{\langle}
\newcommand{\ra}{\rangle}

% Jari

\newtheorem{alg}{Algorithm}
% Maths
\newcommand{\N}{\mathbb{N}}
\renewcommand{\abs}[1]{\lvert #1 \rvert}
\renewcommand{\norm}[1]{\lVert #1 \rVert}
\newcommand{\snorm}[1]{\left\lVert #1 \right\rVert}
\newcommand{\deq}{\mathrel{\mathop:}=}
\newcommand{\eqd}{=\mathrel{\mathop:}}
\renewcommand{\grad}{\nabla}
\renewcommand{\d}{\partial}
%English
\newcommand{\cf}{\hbox{{cf.}}\xspace}
\newcommand{\deletia}{\ldots [deletia] \ldots}
\newcommand{\etal}{\hbox{{et al.}}\xspace}
\newcommand{\eg}{\hbox{{e.g.}}\xspace}
\newcommand{\ie}{\hbox{{i.e.}}\xspace}
\newcommand{\scil}{\hbox{{sc.}}\xspace} %scilicet: it is permitted to know
\newcommand{\st}{\hbox{{s.t.}}\xspace}
\newcommand{\wrt}{\hbox{{w.r.t.}}\xspace}
\newcommand{\etc}{\hbox{{etc.}}\xspace}
\newcommand{\viz}{\hbox{{viz.}}\xspace} %videlicet: it is permitted to see
\newcommand{\vs}{\hbox{{vs.}}\xspace}


\newenvironment{proof_sketch}{%
  \renewcommand{\proofname}{Proof Sketch}\proof}{\endproof}



\usepackage[margin=1in]{geometry}
 \usepackage{setspace}
% \setstretch{}
\usepackage{listings}

\lstset { %
    language=C++,
    backgroundcolor=\color{black!5}, % set backgroundcolor
    basicstyle=\footnotesize,% basic font setting
}

\begin{document}

\title{Computing Large Scale Linear Least Squares with SKiLLS}

\author{Zhen Shao}\maketitle

\section{Overview}

SKiLLS (SKetchIng-Linear-Least-Sqaures) is a C++ package for finding solutions to over-determined linear least square problems. SKiLLS uses a modern dimensionality reduction technique called sketching, and is particularly suited for large scale linear least squares where the number of measurements/observations is far greater than the number of variables. 

Mathematically, SkiLLS solves

\begin{align}
\min_{x \in \R^{d}} f(x): = \|Ax - b\|_2^2, \label{eq::problem}
\end{align}
where $A \in \R^{n\times d}$ and $b \in \R^{n}$ given. The matrix $A$ is allowed to be rank-deficient or nearly rank-deficient. 

\section{Use of External Library}

The location of external libraries are coded in a config.mk file, given below. 

\begin{lstlisting}[numbers=left]
# External library Locations
SUITSPARSEROOT ?= /home/shaoz/SuiteSparse
LAPACKROOT ?= /home/shaoz/lapack-3.8.0/
FortranROOT ?= /home/shaoz/intel/lib/intel64
HSLROOT ?= /home/shaoz/hsledits
RCPQR_ROOT ?=/Users/zhen/Downloads/hqrrp/lapack_compatible_sources
FFTW_ROOT?= /home/shaoz/usr/lib
LIBS_MKL = -Wl,--start-group ${MKLROOT}/lib/intel64/libmkl_intel_ilp64.a \
 ${MKLROOT}/lib/intel64/libmkl_sequential.a \
 ${MKLROOT}/lib/intel64/libmkl_core.a -Wl,-liomp5 -lpthread -lm -ldl

# Optimization Level
OPTIMIZATION = -O3
CXX=icc

INCLUDES = -I./include -I./include/impl -I${SUITSPARSEROOT}/include -I${BOOSTROOT} \
-I${FFTW_INCLUDE}
CXXFLAGS = -Wall ${OPTIMIZATION} -march=native -fPIC -DMKL_ILP64 -I${MKLROOT}/include -std=c++11
CXXFLAGS_DEBUG = -Wall -O0 -g -march=native -fPIC -DMKL_ILP64 -I${MKLROOT}/include

\end{lstlisting}

So that it's easy to see the required dependency for the package, and use the package on different systems (I need to run the software on three different machines.). 


\section{Generating a sparse dimensionality reduction matrix in compressed column format}

\begin{lstlisting}[numbers=left, firstnumber=20]
// Generate an array 1:n
void gen_0_to_n(long n, long *returned_array)
{
    for (long i = 0; i<n; i++){
        returned_array[i] = i;
    }
}

// Randomly shuffle an array so that the first k indices are sampled from 1:n without replacement
void select_k_from_n(
    long n /* length of array */, 
    long k /*number of indices to be shuffled*/, 
    long* array_to_be_shuffled)
{
    // input check
    assert(n >= k);
    for (long i=0; i<k; i++){
        long j = ( rand() % (n-i) ) +i;
        long tmp = array_to_be_shuffled[j];
        array_to_be_shuffled[j] = array_to_be_shuffled[i];
        array_to_be_shuffled[i] = tmp;
    }
}

// Generate a hashing matrix, return in the compressed column sparse format
void gen_hashing_matrix(
    long m /* number of rows */, 
    long n /* number of columns */, 
    long nnz_per_column /* number of non-zeros per column */, 
    // output
    long *row_indices , long *col_array, double *values)
{
    long one_to_m[m];
    gen_0_to_n(m, one_to_m);
    for (int i=0; i<n; i++){
        select_k_from_n(m, nnz_per_column, one_to_m);
        for (int j=0; j<nnz_per_column; j++){
            row_indices[nnz_per_column*i + j] = one_to_m[j];
        }
    } // the above loop generate the row indices

    for (long i=0; i<n+1; i++){
        col_array[i] = nnz_per_column*i;
    } // generate the column array

    double scaling = sqrt(nnz_per_column);
    for (long i=0; i<n; i++){
        for (long j =0; j<nnz_per_column; j++){
            values[i*nnz_per_column +j] = 1/scaling* ((rand()%2)*2-1);
        }
    }
}
\end{lstlisting}


\section{Use abstract linear operator concept for a variety of preconditioner format}

\begin{lstlisting}[numbers=left, firstnumber = 72]
// ----------- SPARSE PRE --------------//
template <typename T>
class LinOP_sparse_preconditioner
    : public LinOp<T>
{
 public:
  typedef long idx_t;
  LinOP_sparse_preconditioner( LinOp<T>& A, cholmod_sparse* R, 
  int success=1, double perturb=1e-6)
      : _A(A) , _R(R), _n(_R->ncol), _m(_A.m()), 
        _success(success), _perturb(perturb)
  {
    assert( A.n() == R->nrow );
    tmp = Vec<T>(A.n());
  }
  virtual idx_t const& m() const { return _m;}
  virtual idx_t const& n() const { return _n;}
  
  // overwrite matrix-vector multiplication by using preconditioner               
  virtual void mv( const char trans, const T alpha, const Vec<T> x, const T beta, Vec<T> y )
  {
    if( trans == 'n' )
    {
      tmp = x.copy();
      cs_usolve_cholmod_structure(_R, tmp.data(), _success, _perturb);
      _A.mv( 'n', alpha, tmp, beta, y );
    }
    else
    {
      _A.mv( 't', 1.0, x, 0.0, tmp );
      cs_utsolve_cholmod_structure(_R, tmp.data(), _success, _perturb);
      scal(beta, y);
      axpy(alpha, tmp, y);
    }
  }                
 private:
  Vec_d     tmp;
  LinOp<T>&  _A;
  cholmod_sparse* _R;
  idx_t _n;
  idx_t _m;
  int _success;
  double _perturb;
};

// ---------------------------- DENSE PRE -----------------//
template <typename T>
class LinOP_dense_preconditioner
    : public LinOp<T>
{
 public:
  typedef long idx_t;
  LinOP_dense_preconditioner( LinOp<T>& A, Mat_d& R)
      : _A(A) , _R(R), 
      _up('u'), _trans('t'), _no_trans('n'), _diag('n')
  {
    assert( A.n() == R.m());
    tmp = Vec<T>(A.n());
  }
  virtual idx_t const& m() const { return _A.m();}
  virtual idx_t const& n() const { return _R.n();}
  
  // overwrite matrix-vector multiplication by using preconditioner               
  virtual void mv( const char trans, const T alpha, const Vec<T> x, const T beta, Vec<T> y )
  {

    if( trans == 'n' )
    {
      tmp = x.copy();
      dtrsv(&_up, &_no_trans, &_diag, &_R.n(), _R.data(), &_R.ld(), tmp.data(), &tmp.inc());
      _A.mv( 'n', alpha, tmp, beta, y );
    }
    else
    {
      _A.mv( 't', 1.0, x, 0.0, tmp );
      dtrsv(&_up, &_trans, &_diag, &_R.n(), _R.data(), &_R.ld(), tmp.data(), &tmp.inc());
      scal(beta, y);
      axpy(1.0, tmp, y);
    }
  }                
 private:
  Vec_d     tmp;
  LinOp<T>&  _A;
  Mat_d& _R;
  char _up;
  char _trans; 
  char _no_trans; 
  char _diag;
};

---------- IC PRE ----------------------------//
template <typename T>
class LinOP_ic_preconditioner
    : public LinOp<T>
{
 public:
  typedef long idx_t;
  LinOP_ic_preconditioner( LinOp<T>& A, void* pkeep)
      : _A(A) , _pkeep(pkeep),
      _trans(1), _no_trans(0), _ifail(0)
  {
    // assert( A.n() == R.m()); cannot do the assertion because pkeep is difficult to handle
    tmp = Vec<T>(A.n());
  }
  virtual idx_t const& m() const { return _A.m();}
  virtual idx_t const& n() const { return _A.n();} // assuming preconditioning size match
  
  // overwrite matrix-vector multiplication by using preconditioner               
  virtual void mv( const char trans, const T alpha, const Vec<T> x, const T beta, Vec<T> y )
  {

    if( trans == 'n' )
    {
      // Note IC returns Lower trianglar, so transpose or no transpose is reversed
      hsl_mi35_solve(&_trans, &_A.n(), &_pkeep, x.data() , tmp.data() , &_ifail); 
      _A.mv( 'n', alpha, tmp, beta, y );
    }
    else
    {
      _A.mv( 't', 1.0, x, 0.0, tmp );
      hsl_mi35_solve(&_no_trans, &_A.n(), &_pkeep, tmp.data(), tmp.data(), &_ifail);
      scal(beta, y);
      axpy(1.0, tmp, y);
    }
  }                
 private:
  Vec_d     tmp;
  LinOp<T>&  _A;
  void* _pkeep;
  fint _trans;
  fint _no_trans;
  fint _ifail;
};


// -------------------------- Vanilla LSQR --------------------------//

void lsqr( LinOp<double>& A, const Vec_d b,
           const double tol, const long maxit,
           Vec_d& x, int& flag, long& it, int debug ) 
{
  assert( A.m() == b.n() );
  assert( A.n() == x.n() );

  Vec_d v( A.n() );
  Vec_d w(v.n());
  double alpha;

  // explicitly make sure it is all zero...
  for (long i=0; i<A.n(); i++){
    v(i) = 0;
  }

  Vec_d u = b.copy();
  double beta = nrm2(u);
  if(beta > 0){
    scal(1.0/beta,u);    
    A.mv( 't', 1.0, u, 0.0, v );
    alpha = nrm2(v); // norm of v might be zero!!!
  }

  if (alpha >0){
    scal(1.0/alpha,v);    
    w= v.copy();
  }

  x = 0.0;
  double nrm_ar = alpha*beta;
  if (nrm_ar ==0){
    it=0;
    flag=0; // converge in 0 iteration
    return;
  } 
  
  double phi, rho;
  double cs, sn;
  double phibar = beta;
  double rhobar = alpha;
  double theta;
  
  double nrm_a    = 0.0;
  double nrm_r;
  
  // double nrm_ar_0 = alpha*beta;
   flag =1;
   it = 0;
  for( long k = 0; k < maxit; ++k )
  {
    it++;

    A.mv( 'n', 1.0, v, -alpha, u );
    beta = nrm2(u);

    if (beta>0){
      scal(1.0/beta,u);      
      nrm_a = sqrt( nrm_a*nrm_a + alpha*alpha + beta*beta );
      A.mv( 't', 1.0, u, -beta, v );
      alpha = nrm2(v);
      if (alpha >0){
        scal(1.0/alpha,v);    
      }
    }

    rho    = sqrt( rhobar*rhobar + beta*beta );
    cs     = rhobar/rho;
    sn     = beta/rho;
    theta  = sn*alpha;
    rhobar = -cs*alpha;
    phi    = cs*phibar;
    phibar = sn*phibar;

    axpy( phi/rho, w, x );

    scal( -theta/rho, w );
    axpy( 1.0, v,  w );

    nrm_r  = phibar;
    nrm_ar = phibar*alpha*fabs(cs);

    if( nrm_ar < tol*nrm_a*nrm_r ){
        flag = 0;
        break;
    }

  }
}

// ---------------------- Preconditioned version, sparse preconditioner ------------------------------//
void lsqr( LinOp<double>& A, const Vec_d b,
           const double tol, const long maxit,
           Vec_d& x, int& flag, long& it, cholmod_sparse* R_11, 
           int success, double perturb, int debug) // LSQR with preconditioner
{
  LinOP_sparse_preconditioner<double> Apre(A, R_11, success, perturb);
  lsqr( Apre, b, tol, maxit, x, flag, it, debug );
}


// ----------------------- Preconditioned lsqr, dense preconditioner ----------------------//

void lsqr_dense_pre( LinOp<double>& A, const Vec_d b,
           const double tol, const long maxit,
           Vec_d& x, int& flag, long& it, Mat_d& R, int debug) // LSQR with preconditioner
{
  LinOP_dense_preconditioner<double> Apre(A, R);
  lsqr( Apre, b, tol, maxit, x, flag, it, debug);
}


------------------------ Preconditioned lsqr, using incomplete cholesky ----------//
void lsqr_ic( LinOp<double>& A, const Vec_d b,
           const double tol, const long maxit,
           Vec_d& x, int& flag, long& it, void* pkeep, int debug) // LSQR with ic preconditioner
{
  LinOP_ic_preconditioner<double> Apre(A, pkeep);
  lsqr( Apre, b, tol, maxit, x, flag, it, debug);
}
\end{lstlisting}

\section{One version of randomised sparse linear least squares solver}

\begin{lstlisting}[numbers=left, firstnumber=329]

// C++ solver for dense linear least squares using hashing
// Solving the linear least sqaures min_x |Ax-b|_2
void ls_dense_hashing_blendenpik( 
    Mat_d & A, /* m by n */
    Vec_d& b, /* m by 1 */
    Vec_d& x, /* solution, n by 1*/
    long& rank, /* detected rank (in case CPQR is used, otherwise equals to n*/
    int& flag, /* LSQR convergence flag (0=not convergent, 1=convergent) */
    long& it, /* LSQR iteration count */
    double gamma, /* over-sampling ratio */
    long k, /* nnz per column in the hashing matrix */
    double abs_tol, /* absolute tolerance for the residual */
    double rcond, /* control minimal diagonal entries of R_11 */
    double it_tol, /* LSQR relative tolerance */
    long max_it, /* LSQR max iteration */
    int debug, /* no use */
    int wisdom /* flag that if fftw wisdom is to be used */
    )
{
    assert( A.m() == b.n() ); 
    assert( A.n() == x.n() );
    if (A.m()<A.n()) {
        throw std::runtime_error( "Matrix is not overdetermined." );
    }
    long n = A.n();
    double t1;
    double t2;
    double t_tmp;

    // sketch
    Vec_d c(ceil(A.n()*gamma)); // RHS to be sketched
    Mat_d As = rand_hashing_dct(A, b, c, gamma, k, wisdom);

    // build preconditioner, test explicit sketching solution
          long* E;
          E = (long*) calloc(As.n(), sizeof(long));
          double *tau = (double*) calloc(n, sizeof(double));
          // --------------- Compute CPQR factorization----------------//
          column_pivoted_qr(As, E, tau);

          // --------------- Compute Q^T c, store the result in c------//
          char left= 'L';
          char right = 'R';
          char trans = 'T';
          char no_trans = 'N';
          long one = 1;
          long two = 2;
          long workspace_size, info;
          double *workspace;
          double wsize_d;

          /* Query workspace size */
          workspace_size = -1;
          workspace = &wsize_d;
          dormqr(&left, &trans, &c.n(), &one, &As.n(), As.data(), &As.ld(), tau, c.data(), &c.n(), \
            workspace, &workspace_size, &info);  
          /* Compute */
          workspace_size = (long)wsize_d;
          workspace = (double *)malloc(sizeof(double) * workspace_size);
          dormqr(&left, &trans, &c.n(), &one, &As.n(), As.data(), &As.ld(), tau, c.data(), &c.n(), \
            workspace, &workspace_size, &info);  

          // ----------------------- get the R and the rank --------//
            rank=0;
            for (long i=0; i< As.n(); i++){
              if (fabs( As(i,i) ) > rcond)
              {
                rank++;
              }
              else
              {
                break;
              }
            }
            Mat_d R;
            R = As.submat(0, rank, 0, rank);  

          // ----------------- Calculate the least square solution by back substitution ---------//
          char up = 'u';
          char diag = 'n';
          dtrsv(&up, &no_trans, &diag, &R.n(), R.data(), &R.ld(), c.data(), &c.inc());  

          // ---------- Get the basic solution ---------------------------- //

          for (long i=0; i < rank; i++){
              x.data()[E[i]-1] = c.data()[i]; 
          }

          for (long i = rank; i<As.n(); i++){
              x.data()[E[i]-1] = 0;
          }  
          // ----------- Test residual ----------------------------- //
          Vec_d rs = b.copy();
          A.mv('n', -1, x, 1, rs);
          std::cout << "Explicit Sketching Residual is: "<<  nrm2(rs) << std::endl;

          if (nrm2(rs) < abs_tol){
            std::cout << "Return solution found by explicit sketching with residual: "<< nrm2(rs) << std::endl;
            it = 0;
            flag = 0;
            return;
          }
    Vec_d xs = x.copy();

    // subselect columns of A
    long forward = 1;
    long backward = 0;
    dlapmt(&forward, &A.m(), &A.n(), A.data(),&A.ld(), E);
    Mat_d Areduced;
    Areduced = A.submat(0, A.m(), 0, rank);
    Vec_d y(rank);

    // preconditioned lsqr (dense preconditioner)
    lsqr_dense_pre(Areduced, rs, it_tol , max_it, y, flag, it, R, debug);
    dtrsv(&up, &no_trans, &diag, &R.n(), R.data(), &R.ld(), y.data(), &y.inc());
    // recover original solution by doing the permutation

    for (long i=0; i < rank; i++){
        x.data()[E[i]-1] = y.data()[i]; 
    }

    for (long i = rank; i<A.n(); i++){
        x.data()[E[i]-1] = 0;
    }

    // add the explicit sketching solution xs (initial guess)
    for (long i=0; i< x.n(); i++){
      x(i) = xs(i) + x(i); 
    }

    // bring A back to its original form
    dlapmt(&backward, &A.m(), &A.n(), A.data(),&A.ld(), E);
    free(E);
}
\end{lstlisting}




\end{document}