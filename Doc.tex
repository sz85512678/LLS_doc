\documentclass[english,11pt]{article}

%PUT ALL THE PREAMBLES HERE
\usepackage[T1]{fontenc}
\usepackage[latin9]{inputenc}

% \makeatletter % to be able to overwrite package containing @
\usepackage{siunitx}
% \usepackage{subfig}
\usepackage{colonequals}
\usepackage[super]{nth}
\usepackage{algorithm}
\usepackage{amsmath}
\usepackage{amsthm}
\usepackage{amssymb}
\usepackage{tikz}
\usepackage{url}
\usepackage{float}
\usepackage{tabularx}
\usepackage{pbox}
\usepackage{tabulary}
\usepackage{breqn}
\usepackage{graphicx}
\usepackage{wrapfig}
\usepackage[font={small}]{caption}
\usepackage{enumitem}
\usepackage{physics}
\usepackage{lipsum}
\usepackage{mwe}
\usepackage{breqn}
\usepackage[noend]{algpseudocode}
\usepackage{subcaption}
\usepackage{graphicx}
\usepackage{enumitem}

\usepackage{hyperref}
\hypersetup{
    colorlinks,
    citecolor=black,
    filecolor=black,
    linkcolor=black,
    urlcolor=black
} % set up hyperlink for table of content
% \usepackage[]{algorithm2e}
% \SetKwInOut{Parameter}{Parameter}

\graphicspath{{./images/}{./}}

\newtheorem{claim}{Claim}
\newtheorem{theorem}{Theorem}
\newtheorem{lemma}{Lemma}
\newtheorem{remark}{Remark}
\newtheorem{corallary}{Corollary}
\newtheorem{conjecture}{Conjecture}
\newtheorem*{observation}{Observation}
\newtheorem*{myQuote}{Quote}
\newtheorem{fact}{Fact}
\newtheorem{definition}{Definition}
\newtheorem{assumption}{Assumption}
\newtheorem{example}{Example}
% \newtheorem{proof}{Proof}

\newcommand{\R}{\mathbb{R}}
\newcommand{\E}{\mathbb{E}}
\renewcommand{\P}{\mathbb{P}}
\newcommand{\I}{\mathbb{I}}

% \renewenvironment{proof}{{\bfseries Proof}}{\qed}

\makeatletter
\def\BState{\State\hskip-\ALG@thistlm}
\makeatother

\newtheorem*{theorem*}{Theorem}
\newcommand*{\vertbar}{\rule[-1ex]{0.5pt}{2.5ex}}
\newcommand*{\horzbar}{\rule[.5ex]{2.5ex}{0.5pt}}

\newcommand{\norms}[1]{\|#1\|_2^2}

\DeclareMathOperator*{\argmax}{arg\,max}
\DeclareMathOperator*{\argmin}{arg\,min}

\renewcommand{\norms}[1]{\|#1\|_2^2}
\newcommand{\la}{\langle}
\newcommand{\ra}{\rangle}

% Jari

\newtheorem{alg}{Algorithm}
% Maths
\newcommand{\N}{\mathbb{N}}
\renewcommand{\abs}[1]{\lvert #1 \rvert}
\renewcommand{\norm}[1]{\lVert #1 \rVert}
\newcommand{\snorm}[1]{\left\lVert #1 \right\rVert}
\newcommand{\deq}{\mathrel{\mathop:}=}
\newcommand{\eqd}{=\mathrel{\mathop:}}
\renewcommand{\grad}{\nabla}
\renewcommand{\d}{\partial}
%English
\newcommand{\cf}{\hbox{{cf.}}\xspace}
\newcommand{\deletia}{\ldots [deletia] \ldots}
\newcommand{\etal}{\hbox{{et al.}}\xspace}
\newcommand{\eg}{\hbox{{e.g.}}\xspace}
\newcommand{\ie}{\hbox{{i.e.}}\xspace}
\newcommand{\scil}{\hbox{{sc.}}\xspace} %scilicet: it is permitted to know
\newcommand{\st}{\hbox{{s.t.}}\xspace}
\newcommand{\wrt}{\hbox{{w.r.t.}}\xspace}
\newcommand{\etc}{\hbox{{etc.}}\xspace}
\newcommand{\viz}{\hbox{{viz.}}\xspace} %videlicet: it is permitted to see
\newcommand{\vs}{\hbox{{vs.}}\xspace}


\newenvironment{proof_sketch}{%
  \renewcommand{\proofname}{Proof Sketch}\proof}{\endproof}



\usepackage[margin=1in]{geometry}
 \usepackage{setspace}
% \setstretch{}
\usepackage{listings}

\lstset { %
    language=C++,
    backgroundcolor=\color{black!5}, % set backgroundcolor
    basicstyle=\footnotesize,% basic font setting
}

\begin{document}

\title{Computing Large Scale Linear Least Squares with SKiLLS}

\author{Zhen Shao}\maketitle

\section{Overview}

SKiLLS (SKetchIng-Linear-Least-Sqaures) is a C++ package for finding solutions to over-determined linear least square problems. SKiLLS uses a modern dimensionality reduction technique called sketching, and is particularly suited for large scale linear least squares where the number of measurements/observations is far greater than the number of variables. 

Mathematically, SkiLLS solves

\begin{align}
\min_{x \in \R^{d}} f(x): = \|Ax - b\|_2^2, \label{eq::problem}
\end{align}
where $A \in \R^{n\times d}$ and $b \in \R^{n}$ given. The matrix $A$ is allowed to be rank-deficient or nearly rank-deficient. 

\subsection{When to use SKiLLS}

If $A$ in \eqref{eq::problem} is dense, the state-of-the-art sketching solver is Blendenpik \cite{doi:10.1137/090767911}. Comparing to solvers in the classical state-of-the-art numerical package LAPACK \cite{laug}, Blendenpik is two times faster on matrices of size $20,000 \times 500$ and four times faster on matrices of size $100,000 \times 2500$. Comparing to classical iterative solver LSQR \cite{Paige:1982aa}, Blendenpik is 80 times faster on matrices of size $20,000 \times 1,000$ with condition number $100$. 

Blendenpik only solves \eqref{eq::problem} when the matrix $A$ has full numerical rank, as illustrated in Table \ref{tab::rank_def_accuracy}. Two solvers are included in the library SKiLLS for dense $A$. The robust version, \\{\it{ls_dense_hashing_blendenpik} }solves problems \ref{eq::problem} when $A$ is numecially rank-deficient but is as fast as Blendenpik when $A$ has full rank (takes 70\% to 130\% time on matrices of different sizes). The fast version, {\it{ls_dense_hashing_blendenpik_noCPQR}} is 1.6 times faster than Blendenpik on coherent matrices\footnote{Matrices of the form $A \in \R^{n\times d} = \bigl(\begin{smallmatrix} I_{d\times d} \\ 0  \end{smallmatrix}\bigr) + 10^{-8}J_{n,d} $ where $J_{n,d} \in \R^{n \times d}$ is a matrix of all ones.} of size $40,000 \times 4,000$ and $90,000 \times 2250$, and slightly faster than Blendenpik on other types of input (about 1.1 times faster).

If $A$ in \eqref{eq::problem} is sparse, the state-of-the-art solvers are SPQR \cite{Davis:2011ft} and HSL \cite{Scott:2014iq}, which uses sparse QR factorization of $A$ and LSQR with an incomplete Cholesky preconditioner, respectively. Our library has a single sparse solver, {\it ls_sparse_spqr}. It significantly outperforms both state-of-the-art sparse solvers on large random sparse ill-conditioned matrices. It is 10 times faster than HSL and 7 times faster than SPQR on $120,000 \times 5,000$ random sparse matrices with $1\%$ non-zero entries and condition number equals to $10^6$. And it is 3 times faster than both HSL and SPQR on $40,000 \times 2,000$ matrices with $1\%$ non-zero entries and condition number equals to $10^6$.

Our sparse solver also performs extremely well on highly over-determined sparse inputs from real applications. It is the fastest solver on more than $75 \%$ of problems in the Florida Matrix Collection \cite{Davis_2011} with the matrix $A$ defined in \eqref{eq::problem} having $n\geq 30d$ and $n \geq 20,000$. It is also the fastest solver on more than $90\%$ of problems with the matrix $A$ having $n\geq 10d$, $n \geq 20,000$ but additionally having more than $1\%$ of non-zero entries. 

Because sketching exploits redundancy of rows, the performance gain compared to non-sketching solvers\footnote{LAPACK, HSL, SPQR} will be larger on $A$ with larger $n/d$ radio.

For further details of computational advantages, see our paper. {\it{Computing Large Scale Linear Least Squares with SKiLLS, C. Cartis and Z. Shao}}.


% Problem \eqref{eq::problem} frequently appears as a sub-problem in many computational and data science problems, for example, in many non-linear/general function minimisation routine linearises the problem and iteratively solve problems of the form \eqref{eq::problem}. In the following situations using SKiLLS yields significant computational advantage comparing to state-of-the-arts\footnote{Blendenpik \cite{doi:10.1137/090767911} for dense problems, SPQR \cite{Davis:2011ft} and Cholesky-preconditioned LSQR \cite{Scott:2014iq} for sparse problems}:

% {\color{red} Begin with LAPACK, and give individual data point about how faster we are comparing to Blendenpik, set expectations about the smallest dimensions and over-determinancy ratio. }
% \begin{enumerate}\setlength{\itemsep}{-2pt}
% 	\item $A$ is dense, large, sufficiently over-determined and (approximately) rank-deficient or with unknown rank. E.g. one of the matrices we ran numerical test on has dimension $50000 \times 4000$.

% 	\item $A$ is dense, large, sufficiently over-determined with high coherence\footnote{The coherence of a matrix A is defined as the largest Euclidean norm of $U$, where $U$ is defined in the reduced singular value decomposition of $A$. }.

% 	\item $A$ is sparse, large, sufficiently over-determined with non-zero entries randomly distributed. 

% 	\item $A$ is moderately sparse (e.g. one percent of entries are non-zero), large and moderately over-determined. 

% \end{enumerate}

% In the following situations using SKiLLS is competitive with state-of-the-arts

% \begin{enumerate}\setlength{\itemsep}{-2pt}
% 	\item $A$ is dense, large, sufficiently over-determined with full rank. 

% 	\item $A$ is sparse, large, moderately over-determined. 
% \end{enumerate}

% Table \ref{tab::rank_def_accuracy} shows that SKiLLS is more robust than Blendenpik when the matrix $A$ is approximately rank-deficient. Figure \ref{fig::compare_blen_coherent} shows that SKiLLS outperforms Blendenpik when the matrix $A$ is (full rank) with high coherence. Figrue \ref{fig::compare_blen_semi-coherent} and Figure \ref{fig::compare_blen_incoherent} shows that SKiLLS is competitive with Blendenpik when the matrix $A$ is (full rank). 

% Figure \ref{fig::sparse_rand} shows that SKiLLS runs more than 10 times faster than current state-of-the-art sparse solvers on random sparse matrices. Figure \ref{fig::density001} shows that SKiLLS significantly outperforms its competitors when $A$ is moderately sparse, large and moderately over-determined.  Figure \ref{fig::all_solver_30}, Figure \ref{fig::all_solver_10_LSQR_5} and Figure \ref{fig::all_solver_10} showcase the competitiveness of SKiLLS on sparse, large and moderately over determined problems. 


\section{Installing SKiLLS}
\subsection{Dependency}
\subsection{Installation instruction}
\subsection{Test the installation}

\section{Using SKiLLS}

This is a library of three routines for solving problem \eqref{eq::problem}. 

\begin{itemize}
	\item If the matrix $A$ is dense, {\it ls_dense_hashing_blendenpik} uses LSQR with a preconditioner built from a sketch of the matrix $A$ using Hashed-Randomised-Hadamard-Transform (HRHT) and Randomised-Column-Pivoted-QR (RCPQR) to solve \eqref{eq::problem}.

	\item If $A$ is dense and has full numerical rank, {\it ls_dense_hashing_blendenpik_noCPQR} uses LSQR with a preconditioner built from a sketch of the matrix $A$ using HRHT and Column-Pivoted-QR (CPQR) to solve \eqref{eq::problem}.

	\item   If $A$ is sparse, {\it ls_sparse_spqr} uses LSQR with a preconditioner built from a sketch of the matrix $A$ using $s-$hashing and sparse QR (SPQR) to solve \eqref{eq::problem}.
\end{itemize}

\subsection{Data structure}

\paragraph{Dense Vector and Matrix}
SKiLLS uses C++ class Vec for dense vectors and C++ class Mat for dense matrices. 
To create a $n\times 1$ vector from already existed data, take a pointer to the data and call Vec(n, *data). 
To create a $n \times d$ matrix from already existed data, take a pointer to the data and call Mat(n, d, *data). Basic matrix operations such as return a specific row, column, or matrix-matrix and matrix vector multiplications are implemented.  
These classes are defined in include/Vec.hpp and include/impl/Mat.hpp respectively. 


\paragraph{Sparse Matrix}
SKiLLS uses Compressed Column Data C structure for sparse matrices and vectors. The sparse data structure is from SuiteSparse \cite{10.1145/2049662.2049670}. The cs_dl and cholmod_sparse structures are defined in cs.h and cholmod_core.h respectively the in SuiteSparse/include folder. 



% In particular, the cs_dl structure used as the input matrix format for sparse linear least squares is taken from the CXSparse package from SuiteSparse. It is a C structure with 7 fields: 

% \begin{itemize}
% 	\setlength\itemsep{-0.5em}
% 	\item nzmax: maximum number of non-zeros. 

% 	\item m: number of rows. 

% 	\item n: number of columns. 

% 	\item *p: column indices (size nzmax)

% 	\item *i: row indices, size nzmax. 

% 	\item *x: numerical values of non-zeros, size nzmax. 

% 	\item nz: -1 for compressed column format. 
% \end{itemize}

% Sometimes we will also use the data structure cholmod_sparse for sparse matrix in compressed column format. It has a similar structures. For more information, see the header file or documentation of SuiteSparse.

% \paragraph{Examples}

% We give two examples of constructing dense and sparse matrices from user given data. 

% The below code snippet creates a dense matrix and a vector.

% \begin{lstlisting}
% 	long m = 4;
% 	long n = 2;
% 	double data_A[8] = {0.306051,-1.53078,1.64493,-1.61322,
% 		-0.2829,0.474476,-0.586278,-0.610202};

% 	double data_b[4] = {0.0649338,0.845946,-0.0164085,0.247119};

% 	Mat_d A(m,n,data_A);
% 	Vec_d b(m,data_b);
% \end{lstlisting}

% The below code snippet creates a sparse matrix.

% \begin{lstlisting}
%     long m = 4;
%     long n = 3;
%     long nnz = 5;

%     long col[4] = {1,  2,  4,  6};
%     long row[5] = {1,  1,  3,  2,  4}; 
%     double val[5] = {2.0,  3.0,  1.0,  4.0,  5.0};
%     long col_0[4];
%     long row_0[5];

%     // convert to zero based indices
%     for (long i=0; i<n+1; i++){
%      col_0[i] = col[i]-1;
%     }

%     for (long i =0; i<nnz; i++){
%      row_0[i] = row[i]-1;
%     }

%     cholmod_common Common, *cc;
%     cc = &Common;
%     cholmod_l_start(cc);
%     cc->print = 4;

%     cholmod_sparse* A;
%     A = cholmod_l_allocate_sparse(
%     m, n, nnz, true, true, 0, CHOLMOD_REAL, cc);

%     A->p = col_0;
%     A->i = row_0;
%     A->x = val;

%     cholmod_l_print_sparse( A, "A", cc);

%     cholmod_l_finish(cc);
% \end{lstlisting}





\subsection{Dense problems}
% // C++ solver for dense linear least squares using hashing
% // Solving the linear least sqaures min_x |Ax-b|_2

To compute a solution of a linear least square problem where the matrix $A$ is dense, a call of the following form should be made. \\

ls_dense_hashing_blendenpik(A, b, x, rank, flag, it, gamma, k, abs_tol, rcond, it_tol, max_it, debug, wisdom):
	\begin{itemize}
	\setlength\itemsep{-0.5em}
	\item Inputs 
		\begin{enumerate}
			\item $\mathbf A$ is type Mat_d containing a $\R^{m\times n}$ matrix defined in \eqref{eq::problem}.
			\item $\mathbf b$ is type Vec_d containing a $\R^m$ vector defined in \eqref{eq::problem}.
		\end{enumerate}
	
	\item Main Outputs
		\begin{enumerate}
			\item { $\mathbf x$} is type Vec_d containing a $\R^n$ vector which is a solution of \eqref{eq::problem}.
		\end{enumerate}

	\item Auxillary Outputs
		\begin{enumerate}
			\item {\bf rank} is a scalar containing the detected numerical rank of the matrix $A$. 
			\item {\bf flag} is a scaler. flag=0 indicates LSQR has converged. flag=1 indicates LSQR has not converged. 
			\item {\bf it} is a scaler indicating the number of LSQR iterations taken. 
		\end{enumerate}
		

	\item Parameters
		\begin{enumerate}
			\item {\bf gamma} is the over-sampling ratio used in sketching. Bigger gamma typically gives higher quality preconditioner but slower running time. The default value of gamma is $1.7$.

			\item $\mathbf k$ is number of non-zeros per column in the hashing matrix as part of sketching. Bigger $k$ typically gives higher quality preconditioner but slower running time. The default value of $k$ is 1. 

			\item {\bf abs_tol} specifies an absolute residual tolerance for the solution $x$ of \ref{eq::problem}. If the algorithm finds an $x$ satisfies $\|Ax-b\|_2 \leq abs_tol$, it terminates and returns $x$. The default value is $10^{-8}$

			\item {\bf it_tol} specifies the relative residual tolerance for LSQR convergence, see \cite{Paige:1982aa}. The default value is $10^{-6}$.

			 \item {\bf max_it} specifies the maximum iteration of LSQR, the default value is $10,000$. 


			 \item {\bf debug} is a flag. If debug=1, the solver prints additional outputs for diagosis. 

			 \item {\bf wisdom} is a flag. If wisdom=1, the macro variable FFTW_WISDOM_FILE in inclde/Config.hpp must be defined as the path to a FFTW wisdom file, see fftw documentation at fftw.org. If wisdom=0, the solver does not use FFTW wisdom and may run slower. 
		\end{enumerate}
		

	\end{itemize}


If the matrix $A$ is known to be of full numerical rank, then a different routine can be used which is slightly faster:

ls_dense_hashing_blendenpik_noCPQR(A, b, x, rank, flag, it, gamma, k, it_tol, max_it, debug, wisdom). The arguments usage is the same as above, but we don't need the rank-detection parameter rcond, and the shortcut related to abs_tol is not implemented.

\paragraph{Example}

The following program solves an example of problem \eqref{eq::problem} with given $A$ and $b$. 

\begin{lstlisting}
#include <iostream>
#include "Config.hpp"
#include "SpMat.hpp"
#include "Random.hpp"
#include <math.h>
#include "RandSolver.hpp"
#include "cs.h"
#include <stdio.h> 
#include "SuiteSparseQR.hpp"
#include "cholmod.h"
#include "lsex_all_in_c.h"
#include "IterSolver.hpp"
#include <sys/timeb.h>
#include "bench_config.hpp"
#include "blendenpik.hpp"

int main(int argc, char **argv)
{
	// create data
	long m = 4;
	long n = 2;
	double data_A[8] = {0.306051,-1.53078,1.64493,-1.61322,
	-0.2829,0.474476,-0.586278,-0.610202};

	double data_b[4] = {0.0649338,0.845946,-0.0164085,0.247119};

	Mat_d A(m,n,data_A);
	Vec_d b(m,data_b);

	// parameters
	long k = NNZ_PER_COLUMN;
	double gamma = OVER_SAMPLING_RATIO;
	long max_it = MAX_IT;
	double it_tol = IT_TOL;
	double rcond = 1e-10;
	int wisdom = 0;
	double abs_tol = ABS_TOL;

	// storage

	long it;
	int flag;
	long rank;
	double residual;   
	Vec_d x(A.n());

	// solve
	ls_dense_hashing_blendenpik(
	A, b, x, rank, flag, it, gamma, k, abs_tol, 
	rcond, it_tol, max_it, debug, wisdom);

	std::cout << "A is: "<< A << std::endl;
	std::cout << "b is: "<< b << std::endl;
	std::cout << "solution is: "<< x << std::endl;

	// compute residual
	A.mv('n', 1, x, -1, b); 
	std::cout << "Residual of ls_blendenpik_hashing is: "<<  nrm2(b) << std::endl;
	std::cout << "Iteration of ls_blendenpik_hashing is: " << it << std::endl;

}
\end{lstlisting}


\subsection{Sparse problems}

To compute a solution of a linear least square problem where the matrix $A$ is sparse, a call of the following form should be made. \\

ls_sparse_spqr(*A, b, x, 
    rank, flag, it, gamma, k, abs_tol, ordering, it_tol, max_it, rcond, peturb, debug):

	\begin{itemize}
	\setlength\itemsep{-0.5em}
	\item Inputs 
		\begin{enumerate}
			\item $\mathbf A$ is pointer to type cs_dl containing a $\R^{m\times n}$ sparse matrix defined in \eqref{eq::problem}.
			\item $\mathbf b$ is type Vec_d containing a $\R^m$ vector defined in \eqref{eq::problem}.
		\end{enumerate}
	
	\item Main Outputs
		\begin{enumerate}
			\item {$\mathbf x$} is type Vec_d containing a $\R^n$ vector which is a solution of \eqref{eq::problem}.
		\end{enumerate}

	\item Auxillary Outputs
		\begin{enumerate}
			\item {\bf rank} is a scalar containing the detected numerical rank of the matrix $A$. 
			\item {\bf flag} is a scaler. flag=0 indicates LSQR has converged. flag=1 indicates LSQR has not converged. 
			\item {\bf it} is a scaler indicating the number of LSQR iterations taken. 
		\end{enumerate}
		

	\item Parameters
		\begin{enumerate}
			\item {\bf gamma} is the over-sampling ratio used in sketching. Bigger gamma typically gives higher quality preconditioner but slower running time. The default value of gamma is $1.7$.

			\item $\mathbf k$ is number of non-zeros per column in the hashing matrix as part of sketching. Bigger $k$ typically gives higher quality preconditioner but slower running time. The default value of $k$ is 1. 

			\item {\bf abs_tol} specifies an absolute residual tolerance for the solution $x$ of \ref{eq::problem}. If the algorithm finds an $x$ satisfies $\|Ax-b\|_2 \leq abs_tol$, it terminates and returns $x$. The default value is $10^{-8}$

			\item {\bf ordering} is an integer specifying a fill-reduced ordering for sparse QR factorization. The default value is {SPQR_ORDERING_DEFAULT}. See SuiteSparseQR documentation for more information. 

			\item {\bf it_tol} specifies the relative residual tolerance for LSQR convergence, see \cite{Paige:1982aa}. The default value is $10^{-6}$.

			\item {\bf max_it} specifies the maximum iteration of LSQR, the default value is $10,000$. 

			\item {\bf rcond} is numerical-ill-conditioning tolerance for the preconditioner returned by sparse QR. If the condition number of the preconditioner is larger than $1/rcond$, a warning will be printed. The default value of rcond is $10^{-12}$

			\item {\bf perturb} is a real number. It has no-effect on the algorithm and is part of an ongoing development. 

			\item {\bf debug} is a flag. If debug=1, the solver prints additional outputs for diagosis. 
		\end{enumerate}
		

	\end{itemize}

\paragraph{Example}

The following program solves an example of problem \eqref{eq::problem} with given sparse $A$ and dense $b$. 

\begin{lstlisting}
#include <iostream>
#include "Config.hpp"
#include "SpMat.hpp"
#include "Random.hpp"
#include <math.h>
#include "RandSolver.hpp"
#include <time.h>
#include "cs.h"
#include <stdio.h> 
#include "SuiteSparseQR.hpp"
#include "cholmod.h"
#include "lsex_all_in_c.h"
#include "IterSolver.hpp"
#include <sys/timeb.h>
#include "bench_config.hpp"

int main(int argc, char const *argv[])
{
	// Create a sparse matrix in compressed column format
    long m = 4;
    long n = 3;
    long nnz = 5;

    long col[4] = {0,  1,  3,  5};
    long row[5] = {0,  0,  2,  1,  3}; 
    double val[5] = {2.0,  3.0,  1.0,  4.0,  5.0};

    cholmod_common Common, *cc;
    cc = &Common;
    cholmod_l_start(cc);
    cc->print = 4;

    cholmod_sparse* A;
    A = cholmod_l_allocate_sparse(
    m, n, nnz, true, true, 0, CHOLMOD_REAL, cc);

    A->p = col;
    A->i = row;
    A->x = val;
    cholmod_l_print_sparse( A, "A", cc);
    cholmod_l_finish(cc);


	// input
	cs_dl *A;
    A = to_cs_dl(Acholmod);

    double *b_data = (double*)malloc((A->m)*sizeof(double));
    for (long i=0; i<A->m; i++){
        b_data[i] = 1;
    }
    Vec_d b(A->m, b_data);

    // parameters
    long k = NNZ_PER_COLUMN;
    double gamma = OVER_SAMPLING_RATIO;
    long max_it = MAX_IT;
    double it_tol = IT_TOL;
    double abs_tol = ABS_TOL;
    double rcond = 1e-12;
    int ordering = SPQR_ORDERING;

    // storage

    long it;
    int flag;
    long rank;
    double t_start;
    double t_finish;
    double residual;   
    Vec_d x_ls_qr(A->n);

    // solve
    ls_sparse_spqr(*A, b, x_ls_qr, 
    rank, flag, it, gamma, k, abs_tol, ordering, it_tol, max_it, RCOND_THRESHOLD, PERTURB, debug);

    CSC_Mat_d A_CSC(A->m, A->n, A->nzmax, 
        (long*)A->i, (long*)A->p, (double*)A->x);
    A_CSC.mv('n', 1, x_ls_qr, -1, b);

    std::cout << "Residual of ls_sparse_spqr is: "<<  nrm2(b) << std::endl;
	return 0;
}
\end{lstlisting}

\bibliography{bib/lib_20Aug.bib} 
\bibliographystyle{abbrv}

\appendix

\section{Numerical Results}
\begin{table}[H]
\scriptsize
\centering
\begin{tabular}{l|rrrrrrr}
               & lp\_ship12l                          & Franz1               & GL7d26                                & cis-n4c6-b2          & lp\_modszk1                           & rel5                                  & ch5-5-b1              \\ 
\hline
SVD          & 18.336                               & 26.503               & 50.875                                & 6.1E-14              & 33.236                                & 14.020                                & 7.3194                \\
Blendenpk      & \multicolumn{1}{l}{~~~~~~~~~NaN~~~~} & 9730.700             & \multicolumn{1}{l}{~~~~~~~~~~NaN~~~~} & 3.0E+02              & \multicolumn{1}{l}{~~~~~~~~~~NaN~~~~} & \multicolumn{1}{l}{~~~~~~~~~~NaN~~~~} & 340.9200              \\
Ski-LLS-dense  & 18.336                               & 26.503               & 50.875                                & 5.3E-14              & 33.236                                & 14.020                                & 7.3194                \\
               & \multicolumn{1}{l}{}                 & \multicolumn{1}{l}{} & \multicolumn{1}{l}{}                  & \multicolumn{1}{l}{} & \multicolumn{1}{l}{}                  & \multicolumn{1}{l}{}                  & \multicolumn{1}{l}{}  \\
               & n3c5-b2                              & ch4-4-b1             & n3c5-b1                               & n3c4-b1              & connectus                             & landmark                              & cis-n4c6-b3           \\ 
\hline
SVD            & 9.0E-15                              & 4.2328               & 3.4641                                & 1.8257               & 282.67                                & 1.1E-05                               & 30.996                \\
Blendenpk      & 1.3E+02                              & 66.9330              & 409.8000                              & 8.9443               & \multicolumn{1}{l}{~~~~~~~~~~NaN~~~~} & \multicolumn{1}{l}{~~~~~~~~~~NaN~~~~} & 3756.200              \\
Ski-LLS-dense  & 5.2E-15                              & 4.2328               & 3.4641                                & 1.8257               & 282.67                                & 1.1E-05                               & 30.996               
\end{tabular}
\caption{Residual of solvers for a range of approximate rank-deficient problems taken from the Florida matrix collection \cite{10.1145/2049662.2049663}. We see while Blendenpik struggles, SKiLLS achives the same residual accuracy as SVD method.}
\label{tab::rank_def_accuracy}
\end{table}





\end{document}